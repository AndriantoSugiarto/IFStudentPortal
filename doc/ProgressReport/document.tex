\documentclass[a4paper,twoside]{article}
\usepackage[T1]{fontenc}
\usepackage[bahasa]{babel}
\usepackage{graphicx}
\usepackage{graphics}
\usepackage{float}
\usepackage[cm]{fullpage}
\pagestyle{myheadings}
\usepackage{etoolbox}
\usepackage{setspace} 
\usepackage{lipsum} 
\usepackage{url}
\setlength{\headsep}{30pt}
\usepackage[inner=2cm,outer=2.5cm,top=2.5cm,bottom=2cm]{geometry} %margin
% \pagestyle{empty}

\makeatletter
\renewcommand{\@maketitle} {\begin{center} {\LARGE \textbf{ \textsc{\@title}} \par} \bigskip {\large \textbf{\textsc{\@author}} }\end{center} }
\renewcommand{\thispagestyle}[1]{}
\markright{\textbf{\textsc{Laporan Perkembangan Pengerjaan Skripsi\textemdash Sem. Ganjil 2015/2016}}}

\onehalfspacing
 
\begin{document}

\title{\@judultopik}
\author{\nama \textendash \@npm} 

%ISILAH DATA DATA BERIKUT INI:
\newcommand{\nama}{Herfan Heryandi}
\newcommand{\@npm}{2012730012}
\newcommand{\tanggal}{03/10/2015} %Tanggal pembuatan dokumen
\newcommand{\@judultopik}{IT STUDENT PORTAL: PEMANFAATAN \textit{WEB SCRAPING} UNTUK KUSTOMISASI PORTAL AKADEMIK MAHASISWA} % Judul/topik anda
\newcommand{\kodetopik}{PAS3904}
\newcommand{\jumpemb}{1} % Jumlah pembimbing, 1 atau 2
\newcommand{\pembA}{Pascal Alfadian}
\newcommand{\pembB}{-}
\newcommand{\semesterPertama}{39 - Ganjil 15/16} % semester pertama kali topik diambil, angka 1 dimulai dari sem Ganjil 96/97
\newcommand{\lamaSkripsi}{1} % Jumlah semester untuk mengerjakan skripsi s.d. dokumen ini dibuat
\newcommand{\kulPertama}{Skripsi 1} % Kuliah dimana topik ini diambil pertama kali
\newcommand{\tipePR}{B} % tipe progress report :
% A : dokumen pendukung untuk pengambilan ke-2 di Skripsi 1
% B : dokumen untuk reviewer pada presentasi dan review Skripsi 1
% C : dokumen pendukung untuk pengambilan ke-2 di Skripsi 2

% Dokumen hasil template ini harus dicetak bolak-balik !!!!

\maketitle

\pagenumbering{arabic}

\section{Data Skripsi} %TIDAK PERLU MENGUBAH BAGIAN INI !!!
Pembimbing utama/tunggal: {\bf \pembA}\\
Pembimbing pendamping: {\bf \pembB}\\
Kode Topik : {\bf \kodetopik}\\
Topik ini sudah dikerjakan selama : {\bf \lamaSkripsi} semester\\
Pengambilan pertama kali topik ini pada : Semester {\bf \semesterPertama} \\
Pengambilan pertama kali topik ini di kuliah : {\bf \kulPertama} \\
Tipe Laporan : {\bf \tipePR} -
\ifdefstring{\tipePR}{A}{
			Dokumen pendukung untuk {\BF pengambilan ke-2 di Skripsi 1} }
		{
		\ifdefstring{\tipePR}{B} {
				Dokumen untuk reviewer pada presentasi dan {\bf review Skripsi 1}}
			{	Dokumen pendukung untuk {\bf pengambilan ke-2 di Skripsi 2}}
		}

\section{Detail Perkembangan Pengerjaan Skripsi}
Detail bagian pekerjaan skripsi sesuai dengan rencana kerja/laporan perkembangan terakhir :
	\begin{enumerate}
		\item Melakukan studi mengenai \textit{library} jsoup, Chrome DevTools, Play Framework, dan SIA Models.\\
		{\bf status :} Ada sejak rencana kerja skripsi kecuali SIA Models.\\
		{\bf hasil :} \textit{library} jsoup, Chrome DevTools, Play Framework, dan SIA Models sudah berhasil dipelajari.
		
		\item Melakukan wawancara kepada mahasiswa Program Studi Teknik Informatika untuk mendapatkan informasi
penggunaan Portal Akademik Mahasiswa.\\
		{\bf status :} Ada sejak rencana kerja skripsi.\\
		{\bf hasil :} Wawancara sudah dilakukan kepada 18 mahasiswa Program Studi Teknik Informatika. Bukti-bukti wawancara dapat dilihat pada \url{https://github.com/herfanheryandi/Skripsi/tree/master/draft/Interview/}.

		\item Menganalisis Portal Akademik Mahasiswa.\\
		{\bf status :} Ada sejak rencana kerja skripsi.\\
		{\bf hasil :} Portal Akademik Mahasiswa sudah berhasil dianalisis untuk memenuhi fitur IT Student Portal baik secara fungsi maupun komunikasi.

		\item Mengimplementasikan \textit{web scraping} menggunakan \textit{library} jsoup.\\
		{\bf status :} Ada sejak rencana kerja skripsi.\\
		{\bf hasil :} \textit{Web scraping} sudah diimplementasikan menggunakan \textit{library} jsoup pada bagian login dan pemeriksaan prasyarat mata kuliah.

		\item Membangun perangkat lunak IT Student Portal.\\
		{\bf status :} Ada sejak rencana kerja skripsi.\\
		{\bf hasil :} Perangkat lunak sudah dibangun menggunakan Play Framework. Fitur yang tersedia yaitu prasyarat mata kuliah.

		\item Melakukan eksperimen dan pengujian\\
		{\bf status :} Ada sejak rencana kerja skripsi.\\
		{\bf hasil :} Pengujian baru dilakukan pada fitur prasyarat mata kuliah oleh beberapa mahasiswa. Hasil pemeriksaan prasyarat mata kuliah sudah akurat.

		\item Menulis dokumen skripsi\\
		{\bf status :} Ada sejak rencana kerja skripsi.\\
		{\bf hasil :} Dokumen skripsi sudah ditulis hingga bab 3.

	\end{enumerate}

\section{Pencapaian Rencana Kerja}
Persentase penyelesaian skripsi sampai dengan dokumen ini dibuat dapat dilihat pada tabel berikut :

\begin{center}
  \begin{tabular}{ | c | c | c | c | l | c |}
    \hline
    1*  & 2*(\%) & 3*(\%) & 4*(\%) &5* &6*(\%)\\ \hline \hline
   1   & 14  & 14  &  &  & 13\\ \hline
    2   & 14 & 14  &   &  & 14\\ \hline
    3   & 14  & 14  &  &  & 13\\ \hline
    4   & 14  &   &  14 &  & 6\\ \hline
    5   & 14  &   & 14 &  & 5\\ \hline
    6   & 14 &   & 14  &  & 3\\ \hline
    7   & 16  & 8  & 8 &  {\footnotesize bab 1, bab 2, dan bab 3 di S1} & 7\\ \hline
    Total  & 100  & 50  & 50 &  & 61\\ \hline
                          \end{tabular}										
\end{center}

Keterangan (*)\\
1 : Bagian pengerjaan Skripsi (nomor disesuaikan dengan detail pengerjaan di bagian 5)\\
2 : Persentase total \\
3 : Persentase yang akan diselesaikan di Skripsi 1 \\
4 : Persentase yang akan diselesaikan di Skripsi 2 \\
5 : Penjelasan singkat apa yang dilakukan di S1 (Skripsi 1) atau S2 (skripsi 2)\\
6 : Persentase yang sudah diselesaikan sampai saat ini 

%\section{Kendala yang dihadapi}
%TULISKAN BAGIAN INI JIKA DOKUMEN ANDA TIPE A ATAU C
%Kendala - kendala yang dihadapi selama mengerjakan skripsi :
%\begin{itemize}
%	\item Terlalu banyak melakukan prokratinasi
%	\item Mengalami kesulitan dalam instalasi Play Framework
%	\item Kurang familiar dengan pemrograman Play Framework
%\end{itemize}

\vspace{1cm}
\centering Bandung, \tanggal\\
\vspace{2cm} \nama \\ 
\vspace{1cm}

Menyetujui, \\
\ifdefstring{\jumpemb}{2}{
\vspace{1.5cm}
\begin{centering} Menyetujui,\\ \end{centering} \vspace{0.75cm}
\begin{minipage}[b]{0.45\linewidth}
% \centering Bandung, \makebox[0.5cm]{\hrulefill}/\makebox[0.5cm]{\hrulefill}/2013 \\
\vspace{2cm} Nama: \pembA \\ Pembimbing Utama
\end{minipage} \hspace{0.5cm}
\begin{minipage}[b]{0.45\linewidth}
% \centering Bandung, \makebox[0.5cm]{\hrulefill}/\makebox[0.5cm]{\hrulefill}/2013\\
\vspace{2cm} Nama: \pemB \\ Pembimbing Pendamping
\end{minipage}
\vspace{0.3cm}
}{
% \centering Bandung, \makebox[0.5cm]{\hrulefill}/\makebox[0.5cm]{\hrulefill}/2013\\
\vspace{2cm} Nama: \pembA \\ Pembimbing Tunggal
}
`
\end{document}

