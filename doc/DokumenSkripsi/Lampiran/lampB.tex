\chapter{Kode Program \textit{Model}}

\singlespacing 
\begin{lstlisting}[language=Java,basicstyle=\tiny,caption=JadwalDisplay.java]
package models.display;

import java.util.List;

import models.support.JadwalKuliah;

public class JadwalDisplay {
	private List<JadwalKuliah> jadwalList;
	private JadwalKuliah[][] kuliahCalendar;
	private String[] hariList;
	
	public JadwalDisplay(List<JadwalKuliah> jadwalList){
		this.jadwalList = jadwalList;
		kuliahCalendar = new JadwalKuliah[6][22];
		hariList = new String[]{"Senin","Selasa","Rabu","Kamis","Jumat","Sabtu","Minggu"};
		fillKuliahCalendar();
	}
	
	public String getHariByIndex(int index){
		return this.hariList[index];
	}
	
	public JadwalKuliah getJadwalKuliah(int indexHari, int indexWaktu){
		return kuliahCalendar[indexHari][indexWaktu];
	}
	
	public boolean isKuliahEmpty(){
		return jadwalList.isEmpty();
	}
	
	private void fillKuliahCalendar(){
		if(!jadwalList.isEmpty()){
            for (int i = 0; i < kuliahCalendar.length; i++) {
                for (int j = 0; j < kuliahCalendar[i].length; j++) {
                    kuliahCalendar[i][j] = new JadwalKuliah();
                }
            }
            for (int i = 0; i < jadwalList.size(); i++) {
                JadwalKuliah jdw = jadwalList.get(i);
                int day = dayTranslate(jdw.getHari());
                if(day!=-1){
                	String[] timePair = jdw.getWaktu().split("-");
                    String start = timePair[0];
                    String end = timePair[1];
                    int range = (Integer.parseInt(end.substring(0, 2))- Integer.parseInt(start.substring(0, 2)))*2;
                    int beginIndex = 0;
                    int half = Character.getNumericValue(start.charAt(3));
                    if(half<3){
                        beginIndex = (Integer.parseInt(start.substring(0, 2))-7)*2;
                    }
                    else if(half>=3){
                        beginIndex =((Integer.parseInt(start.substring(0, 2))-7)*2)+1;
                    }
                    int endHalf = Character.getNumericValue(end.charAt(3));
                    if(endHalf>3){
                    	range++;  
                    }
                    for (int j = beginIndex; j < beginIndex+range; j++) {
                    	kuliahCalendar[day][j] = jdw;      	
                    }
                }
            }
        }
	}
	
	private int dayTranslate(String hari){
        int day = -1;
        switch(hari){
            case "Senin": 
                day = 0;
                break;
            case "Selasa": 
                day = 1;
                break;
            case "Rabu": 
                day = 2;
                break;
            case "Kamis": 
                day = 3;
                break;
            case "Jumat": 
                day = 4;
                break;
            case "Sabtu": 
                day = 5;
                break;
            default:
                day = -1;
                break;
        }
        return day;
    }
}
\end{lstlisting}

\singlespacing 
\begin{lstlisting}[language=Java,basicstyle=\tiny,caption=PrasyaratDisplay.java]
	package models.display;

import models.id.ac.unpar.siamodels.MataKuliah;

public class PrasyaratDisplay {
	private MataKuliah mataKuliah;
	private String[] status;
	
	public PrasyaratDisplay(MataKuliah mataKuliah, String[] status){
		this.mataKuliah = mataKuliah;
		this.status = status;
	}
	
	public MataKuliah getMataKuliah(){
		return mataKuliah;
	}
	
	public String[] getStatus(){
		return status;
	}
}
\end{lstlisting}

\singlespacing 
\begin{lstlisting}[language=Java,basicstyle=\tiny,caption=RingkasanDisplay.java]
	package models.display;

import models.id.ac.unpar.siamodels.MataKuliahFactory;


public class RingkasanDisplay {
	private String[] pilWajib;
	private String[] pilWajibLulus;
	private String[] pilWajibBelumLulus;
	private String IPS;
	private String IPK;
	private int sksLulusTotal;
	private int sksLulusSemTerakhir;
	private String semesterTerakhir;
	private final int MIN_LULUS_PIL_WAJIB = 4;
	
	public RingkasanDisplay(String IPS, String IPK, int sksLulusTotal){
		this.IPS = IPS;
		this.IPK = IPK;
		this.sksLulusTotal = sksLulusTotal;
		/*create mata kuliah pilihan wajib*/
		pilWajib = new String[]{"AIF311","AIF312","AIF313","AIF314","AIF315","AIF316","AIF317","AIF318"}; 
	}
	
	public int getMinLulusPilWajib(){
		return this.MIN_LULUS_PIL_WAJIB;
	}
	
	public String getNamaPilWajib(String kode){
		return MataKuliahFactory.createMataKuliah(kode, MataKuliahFactory.UNKNOWN_SKS, MataKuliahFactory.UNKNOWN_NAMA).nama()+"";
	}
	
	public String[] getPilWajibLulus() {
		return pilWajibLulus;
	}

	public void setPilWajibLulus(String[] pilWajibLulus) {
		this.pilWajibLulus = pilWajibLulus;
	}

	public String[] getPilWajibBelumLulus() {
		return pilWajibBelumLulus;
	}


	public void setPilWajibBelumLulus(String[] pilWajibBelumLulus) {
		this.pilWajibBelumLulus = pilWajibBelumLulus;
	}

	public String[] getPilWajib(){
		return this.pilWajib;
	}
	
	public String getIPS(){
		return this.IPS;
	}
	
	public String getIPK(){
		return this.IPK;
	}
	
	public void setDataSemTerakhir(String semTerakhir, int sksLulusSemTerakhir) {
		this.semesterTerakhir = semTerakhir;
		this.sksLulusSemTerakhir = sksLulusSemTerakhir;
	}
	
	public String getSemesterTerakhir(){
		return semesterTerakhir;
	}
	public int getSKSLulusTotal(){
		return this.sksLulusTotal;
	}
	
	public int getSKSLulusSemTerakhir(){
		return this.sksLulusSemTerakhir;
	}
	
	public int getMinSisaSKS(){
		if(sksLulusTotal>=144){
			return 0;
		}
		else{
			return 144-sksLulusTotal;
		}
	}
}
\end{lstlisting}

\singlespacing 
\begin{lstlisting}[language=Java,basicstyle=\tiny,caption=CustomMahasiswa.java]
package models.support;

import java.util.ArrayList;
import java.util.List;

import models.id.ac.unpar.siamodels.Mahasiswa;

public class CustomMahasiswa extends Mahasiswa{
	protected List<JadwalKuliah> jadwalList;
	protected String photoPath;
	
	public CustomMahasiswa(String npm) throws NumberFormatException {
		super(npm);
		this.jadwalList = new ArrayList<JadwalKuliah>();
	}
	
	public String getPhotoPath(){
    	return photoPath;
    }
	
	public void setPhotoPath(String photoPath) {
		this.photoPath = photoPath;
	}
	
	public List<JadwalKuliah> getJadwalList(){
    	return jadwalList;
    }
	
	public void setJadwalList(List<JadwalKuliah> jadwalList){
    	this.jadwalList = jadwalList;
    }

}
\end{lstlisting}

\singlespacing 
\begin{lstlisting}[language=Java,basicstyle=\tiny,caption=Jadwal.java]
package models.support;

import models.id.ac.unpar.siamodels.MataKuliah;

public class Jadwal {
    protected MataKuliah mataKuliah;
    protected char kelas;
    protected String waktu;
    protected String ruang;

    public Jadwal(MataKuliah mataKuliah, char kelas, String waktu, String ruang) {
        this.mataKuliah = mataKuliah;
        this.kelas = kelas;
        this.waktu = waktu;
        this.ruang = ruang;
    }
    
    public Jadwal() {
        
    }
    
    public MataKuliah getMataKuliah() {
        return mataKuliah;
    }

    public char getKelas() {
        return kelas;
    }

    public String getWaktu() {
        return waktu;
    }

    public String getRuang() {
        return ruang;
    }
    
}
\end{lstlisting}

\singlespacing 
\begin{lstlisting}[language=Java,basicstyle=\tiny,caption=JadwalKuliah.java]
	package models.support;

import models.id.ac.unpar.siamodels.MataKuliah;

public class JadwalKuliah extends Jadwal{
    private String dosen;
    private String hari;

    public JadwalKuliah(MataKuliah mataKuliah, char kelas, String dosen, String hari, String waktu, String ruang) {
        super(mataKuliah, kelas, waktu, ruang);
        this.dosen = dosen;
        this.hari = hari;
    }
    
    public JadwalKuliah(){
        super();
    }
        
    public String getDosen() {
        return dosen;
    }

    public String getHari() {
        return hari;
    }

}
\end{lstlisting}

\singlespacing 
\begin{lstlisting}[language=Java,basicstyle=\tiny,caption=Scraper.java]
package models.support;

import java.io.IOException;
import java.util.ArrayList;
import java.util.List;
import java.util.Map;

import models.id.ac.unpar.siamodels.Mahasiswa;
import models.id.ac.unpar.siamodels.Mahasiswa.Nilai;
import models.id.ac.unpar.siamodels.MataKuliah;
import models.id.ac.unpar.siamodels.MataKuliahFactory;
import models.id.ac.unpar.siamodels.Semester;
import models.id.ac.unpar.siamodels.TahunSemester;

import org.jsoup.Connection;
import org.jsoup.Connection.Response;
import org.jsoup.Jsoup;
import org.jsoup.nodes.Document;
import org.jsoup.nodes.Element;
import org.jsoup.select.Elements;

public class Scraper {
    private final String BASE_URL = "https://studentportal.unpar.ac.id/";
    private final String LOGIN_URL = BASE_URL + "home/index.login.submit.php";
    private final String CAS_URL = "https://cas.unpar.ac.id/login";
    private final String ALLJADWAL_URL = BASE_URL + "includes/jadwal.all.php";
    private final String JADWAL_URL = BASE_URL + "includes/jadwal.aktif.php";
    private final String NILAI_URL = BASE_URL + "includes/nilai.sem.php";
    private final String LOGOUT_URL = BASE_URL + "home/index.logout.php";
    private TahunSemester currTahunSemester;
    private List<MataKuliah> mkList;
    
    public List<MataKuliah> getMkList(){
    	return this.mkList;
    }
    
    public String getSemester() {
        return currTahunSemester.getSemester() +" "+currTahunSemester.getTahun()+"/"+(currTahunSemester.getTahun()+1);
    }
    
    public void init() throws IOException{
        Connection baseConn = Jsoup.connect(BASE_URL);
        baseConn.timeout(0);
        baseConn.validateTLSCertificates(false);
        baseConn.method(Connection.Method.GET);
        baseConn.execute(); 
    }
    
    public CustomMahasiswa login(String npm, String pass) throws IOException{
        init();
    	CustomMahasiswa logged_mhs = new CustomMahasiswa(npm);	
        String user = logged_mhs.getEmailAddress();
        Connection conn = Jsoup.connect(LOGIN_URL);
        conn.data("Submit", "Login");
        conn.timeout(0);
        conn.validateTLSCertificates(false);
        conn.method(Connection.Method.POST);
        Response resp = conn.execute();
        Document doc = resp.parse();
        String lt = doc.select("input[name=lt]").val();
        String execution = doc.select("input[name=execution]").val();
        /*CAS LOGIN*/
        Connection loginConn = Jsoup.connect(CAS_URL);
        loginConn.cookies(resp.cookies());
        loginConn.data("username",user);
        loginConn.data("password", pass);
        loginConn.data("lt", lt);
        loginConn.data("execution", execution);
        loginConn.data("_eventId","submit");
        loginConn.timeout(0);
        loginConn.validateTLSCertificates(false);
        loginConn.method(Connection.Method.GET);
        resp = loginConn.execute();
        if(resp.body().contains("Data Akademik")){
        	Map<String,String> login_cookies = resp.cookies();
            doc = resp.parse();
            String nama = doc.select("p[class=student-name]").text();
            logged_mhs.setNama(nama);
            String curr_sem = doc.select(".main-info-semester a").text();
            String[] sem_set = this.parseSemester(curr_sem);
            Element photo = doc.select(".student-photo img").first();
            String photoPath = photo.absUrl("src"); 
            logged_mhs.setPhotoPath(photoPath);
            currTahunSemester = new TahunSemester(Integer.parseInt(sem_set[0]),Semester.fromString(sem_set[1]));
            this.requestKuliah(login_cookies);
            List<JadwalKuliah> jadwalList = this.requestJadwal(login_cookies);
            logged_mhs.setJadwalList(jadwalList);
            this.setNilai(login_cookies, logged_mhs);
            logout();
            return logged_mhs;
        }       
        else{
            return null;
        }
    }
    
    public void requestKuliah(Map<String,String> login_cookies) throws IOException{
        Connection kuliahConn = Jsoup.connect(ALLJADWAL_URL);
        kuliahConn.cookies(login_cookies);
        kuliahConn.timeout(0);
        kuliahConn.validateTLSCertificates(false); 
        kuliahConn.method(Connection.Method.GET);
        Response resp = kuliahConn.execute();
        Document doc = resp.parse();
        Elements jadwal = doc.select("tr");
        String prev = "";
        mkList = new ArrayList<MataKuliah>();
        for (int i = 1; i < jadwal.size()-1; i++) {
            Elements row = jadwal.get(i).children();
            if(!row.get(1).text().equals("")){
                String kode = row.get(1).text();
                String nama = row.get(2).text();
                String sks = row.get(3).text();
                if(!kode.equals(prev)){
                    MataKuliah curr = MataKuliahFactory.createMataKuliah(kode, Integer.parseInt(sks), nama);
                    mkList.add(curr);
                }
                prev = kode;
            }   
        }    
    }
    
    public List<JadwalKuliah> requestJadwal(Map<String,String> login_cookies) throws IOException{
        Connection jadwalConn = Jsoup.connect(JADWAL_URL);
        jadwalConn.cookies(login_cookies);
        jadwalConn.timeout(0);
        jadwalConn.validateTLSCertificates(false); 
        jadwalConn.method(Connection.Method.GET);
        Response resp = jadwalConn.execute();
        Document doc = resp.parse();
        Elements jadwalTable = doc.select(".portal-full-table"); 
        List<JadwalKuliah> jadwalList = new ArrayList<JadwalKuliah>();
        
        /*Kuliah*/
        if(jadwalTable.size()>0){
           Elements tableKuliah = jadwalTable.get(0).select("tbody tr");
           String kode = new String(); 
           String nama = new String();
           for(Element elem : tableKuliah){
               if(elem.className().contains("row")){       
                    if(!(elem.child(1).text().isEmpty() && elem.child(2).text().isEmpty())){
                        kode = elem.child(1).text();
                        nama = elem.child(2).text();  
                    }  
                    MataKuliah currMk = MataKuliahFactory.createMataKuliah(kode, Integer.parseInt(elem.child(3).text()), nama);
                    jadwalList.add(new JadwalKuliah(currMk,elem.child(4).text().charAt(0),elem.child(5).text(),elem.child(7).text(),elem.child(8).text(),elem.child(9).text()));
               }
            }
        }      
        return jadwalList;
    }
    
    public void setNilai(Map<String,String> login_cookies, Mahasiswa logged_mhs) throws IOException{  
        Connection nilaiConn = Jsoup.connect(NILAI_URL);
        nilaiConn.cookies(login_cookies);
        nilaiConn.data("npm",logged_mhs.getNpm());
        nilaiConn.data("thn_akd","ALL");
        nilaiConn.timeout(0);
        nilaiConn.validateTLSCertificates(false); 
        nilaiConn.method(Connection.Method.POST);
        Response resp = nilaiConn.execute();
        Document doc = resp.parse();
        Elements mk = doc.select("table");

        for(Element tb:mk){
            Elements tr = tb.select("tr");
            String[] sem_set = this.parseSemester(tr.get(0).text());
            String thn = sem_set[0];
            String sem = sem_set[1]; 
            
            for(Element td:tr){           
                if(td.className().contains("row")){
                  String kode = td.child(1).text();
                  int sks = Integer.parseInt(td.child(3).text());
                  String nama_mk = td.child(2).text();
                  MataKuliah curr_mk = MataKuliahFactory.createMataKuliah(kode, sks, nama_mk);
                  char kelas = td.child(4).text().charAt(0);
                  double ART = 0;
                  double UTS = 0;
                  double UAS = 0;
                  if(kelas!='*'){
                    ART = Double.parseDouble(td.child(5).text());
                    UTS = Double.parseDouble(td.child(6).text());
                    UAS = Double.parseDouble(td.child(7).text());
                  }  
                  if(!td.child(9).text().equals("")){
                	  char NA = td.child(9).text().charAt(0);
                	  TahunSemester tahunSemesterNilai = new TahunSemester(Integer.parseInt(thn),Semester.fromString(sem));
                      logged_mhs.getRiwayatNilai().add(new Nilai(tahunSemesterNilai, curr_mk, kelas, ART, UTS, UAS, NA));
                  }	    
                }
                
            }
        }
    }
    
    public void logout() throws IOException{
        Connection logoutConn = Jsoup.connect(LOGOUT_URL);
        logoutConn.timeout(0);
        logoutConn.validateTLSCertificates(false);
        logoutConn.method(Connection.Method.GET);
        logoutConn.execute();
    }
    
    private String[] parseSemester(String sem_raw){
         String[] sem_set = sem_raw.split("/")[0].split("-");
         return new String[]{sem_set[1].trim(),sem_set[0].trim()};
    }    
}
\end{lstlisting}