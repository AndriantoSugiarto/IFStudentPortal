\chapter{Kode Program \textit{View}}

\singlespacing 
\begin{lstlisting}[language=html,basicstyle=\tiny,caption=login.scala.html]
 @(message: String)

<!doctype html>
<html>
	<head>
		<meta charset="UTF-8">
		<meta http-equiv="X-UA-Compatible" content="IE=edge">
        <meta name="viewport" content="width=device-width, initial-scale=1">
        <link rel="icon" type="image/x-icon" href="@routes.Assets.versioned("images/logo-IT.png")" />
		<title>Informatika Student Portal</title>

		<!-- CSS -->
        <link rel="stylesheet" href="@routes.Assets.versioned("bootstrap/css/bootstrap.min.css")">
        <link rel="stylesheet" href="@routes.Assets.versioned("stylesheets/main.css")">
        
	</head>
	
	<body id="login-body">
	    <div class="container" id="login-container">
			<div class="row">
				<div class="col-md-4 col-sm-2"></div>
				<div class="col-md-4 col-sm-8">
					<div class="panel panel-default">
						<div class="panel-body">
							<div class="page-header">
								<center>
								<div class="logo-set">	
									<img src="@routes.Assets.versioned("images/logo-unpar.png")" width="120px" height="120px"/>
									<img src="@routes.Assets.versioned("images/logo-IT.png")" width="120px" height="120px"/>
								</div>
								</center>
								<br>
								<h4 class="text-center">INFORMATIKA STUDENT PORTAL</h4> 
								<h5 class="text-center">Portal Akademik Mahasiswa</h5>
								<h5 class="text-center">Teknik Informatika UNPAR</h5>
							</div>
							<form action="@routes.Application.submitLogin()" method="POST" name ="loginForm">
								<div class="form-group">
									<label for="email-input">Email</label> 
									<div class="input-group">
										<input name="email" type="email" class="form-control" id="email-input" placeholder="co: 7312012&#64;student.unpar.ac.id"/>
										<span class="input-group-addon"><span class="glyphicon glyphicon-user"></span>
									</div>
								</div>
								<div class="form-group">
									<label for="pw-input">Password</label> 
									<div class="input-group">
										<input name="pass" type="password" class="form-control" id="pw-input" placeholder="********"/>
										<span class="input-group-addon"><span class="glyphicon glyphicon-lock"></span>
									</div>
								</div>
								 @Html(message)
								<button name="submit" type="submit" class="form-control">Login <span class="glyphicon glyphicon-log-in"></span></button>
							</form>
						</div>	
					</div>										
				</div>
				<div class="col-md-4 col-sm-2"></div>
			</div>
		</div>
		<!-- Javascript -->
		<script src="https://ajax.googleapis.com/ajax/libs/jquery/1.11.3/jquery.min.js"></script>
		<script src="@routes.Assets.versioned("bootstrap/js/bootstrap.min.js")></script>

	</body>
</html>
\end{lstlisting}

\singlespacing 
\begin{lstlisting}[language=html,basicstyle=\tiny,caption=home.scala.html]
@(mhs: models.support.CustomMahasiswa)

<!DOCTYPE html>

<html lang="en">
    <head>
        <meta charset="UTF-8">
		<meta http-equiv="X-UA-Compatible" content="IE=edge">
        <meta name="viewport" content="width=device-width, initial-scale=1">
        <link rel="icon" type="image/x-icon" href="@routes.Assets.versioned("images/logo-IT.png")" />
		<title>Informatika Student Portal</title>

		<!-- CSS -->
        <link rel="stylesheet" href="@routes.Assets.versioned("bootstrap/css/bootstrap.min.css")">
        <link rel="stylesheet" href="@routes.Assets.versioned("stylesheets/main.css")"> 
    </head>
    <body>
		<nav class="navbar navbar-inverse sidebar" role="navigation">
			<div class="container-fluid">
				<!-- Logo dan Toggle -->
				<div class="navbar-header">
					<button type="button" class="navbar-toggle" data-toggle="collapse" data-target="#sidebar-navbar-collapse">
						<span class="icon-bar"></span>
						<span class="icon-bar"></span>
						<span class="icon-bar"></span>
					</button>
					<div class="logo-set">
						<img src="@routes.Assets.versioned("images/logo-unpar.png")" width="25%" height="25%"/>
						<img src="@routes.Assets.versioned("images/logo-IT.png")" width="25%" height="25%"/>
					</div>
					<p class="navbar-brand">INFORMATIKA STUDENT PORTAL</p>
				</div>
				<!-- Menu Navigasi -->
				<div class="collapse navbar-collapse" id="sidebar-navbar-collapse">
					<ul class="nav navbar-nav">
						<li class="active" ><a href="@routes.Application.home()">
							Home<span style="font-size:16px;" class="pull-right hidden-xs showopacity glyphicon glyphicon-home"></span>
						</a></li>
						<li><a href="@routes.Application.prasyarat()">
							Prasyarat Mata Kuliah<span style="font-size:16px;" class="pull-right hidden-xs showopacity glyphicon glyphicon-tasks"></span>
						</a></li>
						<li><a href="@routes.Application.jadwalKuliah()">
							 Jadwal Kuliah<span style="font-size:16px;" class="pull-right hidden-xs showopacity glyphicon glyphicon-calendar"></span>
						</a></li>
						<li><a href="@routes.Application.ringkasan()">
							Data Akademik<span style="font-size:16px;" class="pull-right hidden-xs showopacity glyphicon glyphicon-book"></span>
						</a></li>
						<li ><a href="@routes.Application.logout()">
							Logout<span style="font-size:16px;" class="pull-right hidden-xs showopacity glyphicon glyphicon-log-out"></span>
						</a></li>
					</ul>
				</div>
			</div>
		</nav>
		<div class="main">
		<!-- Konten Halaman -->
			<div class="container-fluid">
				<div class="row">
					<h2>Selamat datang di Informatika Student Portal!</h2>
				</div>
				<p><br></p>
				<div class="row">
                    <div class="col-lg-6">
						<div class="table-responsive profile-table">
							<table class="table table-bordered">
								<tr>
									<td width="50" rowspan="3"><center><img src="@mhs.getPhotoPath()"/></center></td>
									<td><h6>@mhs.getNama()</h6></td>
								</tr>
								<tr>
									<td><h6>@mhs.getNpm()</h6></td>
								</tr>
								<tr>
									<td><h6>@mhs.getEmailAddress()</h6></td>
								</tr>
							</table>
						</div>
					</div>
				</div>
				<p><br></p>
				<div class="row">
					<div class="col-lg-6">
						<a href="https://github.com/herfanheryandi/Skripsi/tree/master/app/StudentPortal" target="_blank">Source code Informatika Student Portal</a>
					</div>
				</div>
			</div>
		</div>

		<!-- Javascript -->
		<script src="@routes.Assets.versioned("javascripts/jquery-1.11.3.min.js")"></script>
		<script src="@routes.Assets.versioned("javascripts/script.js")"></script>
		<script src="@routes.Assets.versioned("bootstrap/js/bootstrap.min.js")"></script>
		
    </body>
</html>
\end{lstlisting}



\singlespacing 
\begin{lstlisting}[language=html,basicstyle=\tiny,caption=prasyarat.scala.html]
@(list: List[models.display.PrasyaratDisplay], semester: String)

<!DOCTYPE html>

<html lang="en">
    <head>
        <meta charset="UTF-8">
		<meta http-equiv="X-UA-Compatible" content="IE=edge">
        <meta name="viewport" content="width=device-width, initial-scale=1">
        <link rel="icon" type="image/x-icon" href="@routes.Assets.versioned("images/logo-IT.png")" />
		<title>Informatika Student Portal</title>

		<!-- CSS -->
        <link rel="stylesheet" href="@routes.Assets.versioned("bootstrap/css/bootstrap.min.css")">
        <link rel="stylesheet" href="@routes.Assets.versioned("stylesheets/main.css")"> 
    </head>
    <body>
		<nav class="navbar navbar-inverse sidebar" role="navigation">
			<div class="container-fluid">
				<!-- Logo dan Toggle -->
				<div class="navbar-header">
					<button type="button" class="navbar-toggle" data-toggle="collapse" data-target="#sidebar-navbar-collapse">
						<span class="icon-bar"></span>
						<span class="icon-bar"></span>
						<span class="icon-bar"></span>
					</button>
					<div class="logo-set">
						<img src="@routes.Assets.versioned("images/logo-unpar.png")" width="25%" height="25%"/>
						<img src="@routes.Assets.versioned("images/logo-IT.png")" width="25%" height="25%"/>
					</div>
					<p class="navbar-brand">INFORMATIKA STUDENT PORTAL</p>
				</div>
				<!-- Menu Navigasi -->
				<div class="collapse navbar-collapse" id="sidebar-navbar-collapse">
					<ul class="nav navbar-nav">
						<li><a href="@routes.Application.home()">
							Home<span style="font-size:16px;" class="pull-right hidden-xs showopacity glyphicon glyphicon-home"></span>
						</a></li>
						<li class="active"><a href="@routes.Application.prasyarat()">
							Prasyarat Mata Kuliah<span style="font-size:16px;" class="pull-right hidden-xs showopacity glyphicon glyphicon-tasks"></span>
						</a></li>
						<li><a href="@routes.Application.jadwalKuliah()">
							 Jadwal Kuliah<span style="font-size:16px;" class="pull-right hidden-xs showopacity glyphicon glyphicon-calendar"></span>
						</a></li>
						<li ><a href="@routes.Application.ringkasan()">
							Data Akademik<span style="font-size:16px;" class="pull-right hidden-xs showopacity glyphicon glyphicon-book"></span>
						</a></li>
						<li ><a href="@routes.Application.logout()">
							Logout<span style="font-size:16px;" class="pull-right hidden-xs showopacity glyphicon glyphicon-log-out"></span>
						</a></li>
					</ul>
				</div>
			</div>
		</nav>
		<div class="main">
		<!-- Konten Halaman -->
			<div class="container-fluid">
				<div class="row">
					<h2 class="text-center">PEMERIKSAAN PRASYARAT MATA KULIAH</h2> 
					<h4 class="text-center">SEMESTER @semester</h4> 
				</div>
				@if(list==null){
					<div class="row">
						<h5 style="color: gray;">PRASYARAT BELUM TERSEDIA <span class="glyphicon glyphicon-exclamation-sign"></span></h5>
					</div>	
				}else{
				<div class="row">
                    <div class="col-lg-12">
						<div class="table-responsive">
						<table class="table table-bordered">
						<thead>
							<tr>
								<th class="text-center" >Kode Mata Kuliah</th>
								<th class="text-center" >Nama Mata Kuliah</th>
								<th class="text-center" >Keterangan</th>
							</tr>
						</thead>
						<tbody>
						@for(ls<-list){
							<tr>
								@if(ls.getStatus()(0).contains("data prasyarat tidak tersedia")){
									<td class="text-center" rowspan="@(ls.getStatus().length)">@ls.getMataKuliah().kode()</td>
								}else{
									<td class="text-center" rowspan="@(ls.getStatus().length)"><a target="_blank" href= @{val ref = "https://github.com/herfanheryandi/Skripsi/blob/master/app/StudentPortal/app/models/id/ac/unpar/siamodels/matakuliah/"+ls.getMataKuliah().kode()+".java"; ref}>@ls.getMataKuliah().kode()</a></td>
								}

								<td rowspan="@(ls.getStatus().length)">@ls.getMataKuliah().nama()</td>
								
								@for(stat<-ls.getStatus()){
									@if(!stat.isEmpty){
										@{
											if(stat.equals(ls.getStatus()(0))){
												if(stat.contains("sudah lulus")){
													<td style="color: #0066FF;">{stat} <span class="glyphicon glyphicon-flag"></span></td>
												}else if(stat.contains("memenuhi syarat")){
													<td style="color: green;">{stat} <span class="glyphicon glyphicon-ok"></span></td>
												}else if(stat.contains("tidak memiliki prasyarat")){
													<td style="color: green;">{stat} <span class="glyphicon glyphicon-ok"></span></td>
												}else if(stat.contains("CATATAN:")){
													<td style="color: #FF9900;">{stat} <span class="glyphicon glyphicon-warning-sign"></span></td>
												}else if(stat.contains("data prasyarat tidak tersedia")){
													<td style="color: gray;">{stat} <span class="glyphicon glyphicon-exclamation-sign"></span></td>
												}else{
													<td style="color: red;">{stat} <span class="glyphicon glyphicon-remove"></span></td>
												}	
											}
											else{
												if(stat.contains("CATATAN:")){
													<tr><td style="color: #FF9900;">{stat} <span class="glyphicon glyphicon-warning-sign"></span></td></tr>
												}else{
													<tr><td style="color: red;">{stat} <span class="glyphicon glyphicon-remove"></span></td></tr>
												}
											}
										}
									}
								}
							</tr>
						}
						</tbody>
						</table>
						</div>
					</div>
                </div>
                }
			</div>
		</div>
		
		<!-- Javascript -->
		<script src="@routes.Assets.versioned("javascripts/jquery-1.11.3.min.js")"></script>
		<script src="@routes.Assets.versioned("javascripts/script.js")"></script>
		<script src="@routes.Assets.versioned("bootstrap/js/bootstrap.min.js")"></script>
    </body>
</html>
\end{lstlisting}



\singlespacing 
\begin{lstlisting}[language=html,basicstyle=\tiny,caption=jadwalKuliah.scala.html]
@(jadwalKuliah: models.display.JadwalDisplay, semester: String)
<!DOCTYPE html>

<html lang="en">
    <head>
        <meta charset="UTF-8">
		<meta http-equiv="X-UA-Compatible" content="IE=edge">
        <meta name="viewport" content="width=device-width, initial-scale=1">
        <link rel="icon" type="image/x-icon" href="@routes.Assets.versioned("images/logo-IT.png")" />
		<title>Informatika Student Portal</title>

		<!-- CSS -->
        <link rel="stylesheet" href="@routes.Assets.versioned("bootstrap/css/bootstrap.min.css")">
        <link rel="stylesheet" href="@routes.Assets.versioned("stylesheets/main.css")"> 
    </head>
    <body>
		<nav class="navbar navbar-inverse sidebar" role="navigation">
			<div class="container-fluid">
				<!-- Logo dan Toggle -->
				<div class="navbar-header">
					<button type="button" class="navbar-toggle" data-toggle="collapse" data-target="#sidebar-navbar-collapse">
						<span class="icon-bar"></span>
						<span class="icon-bar"></span>
						<span class="icon-bar"></span>
					</button>
					<div class="logo-set">
						<img src="@routes.Assets.versioned("images/logo-unpar.png")" width="25%" height="25%"/>
						<img src="@routes.Assets.versioned("images/logo-IT.png")" width="25%" height="25%"/>
					</div>
					<p class="navbar-brand">INFORMATIKA STUDENT PORTAL</p>
				</div>
				<!-- Menu Navigasi -->
				<div class="collapse navbar-collapse" id="sidebar-navbar-collapse">
					<ul class="nav navbar-nav">
						<li ><a href="@routes.Application.home()">
							Home<span style="font-size:16px;" class="pull-right hidden-xs showopacity glyphicon glyphicon-home"></span>
						</a></li>
						<li><a href="@routes.Application.prasyarat()">
							Prasyarat Mata Kuliah<span style="font-size:16px;" class="pull-right hidden-xs showopacity glyphicon glyphicon-tasks"></span>
						</a></li>
						<li class="active"><a href="@routes.Application.jadwalKuliah()">
							 Jadwal Kuliah<span style="font-size:16px;" class="pull-right hidden-xs showopacity glyphicon glyphicon-calendar"></span>
						</a></li>
						<li ><a href="@routes.Application.ringkasan()">
							Data Akademik<span style="font-size:16px;" class="pull-right hidden-xs showopacity glyphicon glyphicon-book"></span>
						</a></li>
						<li ><a href="@routes.Application.logout()">
							Logout<span style="font-size:16px;" class="pull-right hidden-xs showopacity glyphicon glyphicon-log-out"></span>
						</a></li>
					</ul>
				</div>
			</div>
		</nav>
		<div class="main">
		<!-- Konten Halaman -->
			<div class="container-fluid">
				<div class="row">
					<h2 class="text-center">JADWAL KULIAH</h2> 
					<h4 class="text-center">SEMESTER @semester</h4>
				</div>
				@if(jadwalKuliah.isKuliahEmpty()){
					<div class="row">
						<h5 style="color: gray;">JADWAL KULIAH BELUM TERSEDIA <span class="glyphicon glyphicon-exclamation-sign"></span></h5>
					</div>	
				}else{
				<div class="row">
					<div class="col-md-1">
						<table id="timeline" class="table hidden-xs hidden-sm">
							<thead>
								<tr>
									<th>Pukul</th>
								</tr>
							</thead>
							<tbody>	
								@for(i <- 7 until 18) {
									 <tr><td height="75" colspan="2">@(i).00 </td></tr>
								}	
							</tbody>
						</table>
					</div>	
					<div class="col-md-10">
						<div class="row">
						@for(day <- 0 until 6) {
							<div class="col-md-2">
								<table id=@jadwalKuliah.getHariByIndex(day) class="table">
									<thead>
										<tr>
											<th class="text-center" colspan="2">@jadwalKuliah.getHariByIndex(day)</th>
										</tr>
									</thead>
									<tbody>	
										@for(i <- 7 until 18) {
											 <tr>
												<td height="75" rowspan="2" class="hidden-md hidden-lg">@(i).00 </td>
												@if(jadwalKuliah.getJadwalKuliah(day,(i-7)*2).getMataKuliah()!=null){	
													<td class="jadwal-cell text-center" rowspan="1" height="25" data-toggle="modal" data-target="#modalDialog" onclick="setModal('@jadwalKuliah.getJadwalKuliah(day,(i-7)*2).getMataKuliah().kode()','@jadwalKuliah.getJadwalKuliah(day,(i-7)*2).getMataKuliah().nama()','@jadwalKuliah.getJadwalKuliah(day,(i-7)*2).getKelas()','@jadwalKuliah.getJadwalKuliah(day,(i-7)*2).getDosen()','@jadwalKuliah.getJadwalKuliah(day,(i-7)*2).getWaktu()','@jadwalKuliah.getJadwalKuliah(day,(i-7)*2).getRuang()')">@(jadwalKuliah.getJadwalKuliah(day,(i-7)*2).getMataKuliah().kode())</td>
												}else{
													<td class="text-center" height="25"> - </td>
												}
											 </tr>
											 @if(jadwalKuliah.getJadwalKuliah(day,((i-7)*2)+1).getMataKuliah()!=null){	
												<tr><td class="jadwal-cell text-center" height="25" data-toggle="modal" data-target="#modalDialog" onclick="setModal('@jadwalKuliah.getJadwalKuliah(day,((i-7)*2)+1).getMataKuliah().kode()','@jadwalKuliah.getJadwalKuliah(day,((i-7)*2)+1).getMataKuliah().nama()','@jadwalKuliah.getJadwalKuliah(day,((i-7)*2)+1).getKelas()','@jadwalKuliah.getJadwalKuliah(day,((i-7)*2)+1).getDosen()','@jadwalKuliah.getJadwalKuliah(day,((i-7)*2)+1).getWaktu()','@jadwalKuliah.getJadwalKuliah(day,((i-7)*2)+1).getRuang()')">@(jadwalKuliah.getJadwalKuliah(day,((i-7)*2)+1).getMataKuliah().kode())</td></tr>
											 }else{
												<tr><td class="text-center" height="25"> - </td></tr>
											 }
										}	
									</tbody>
								</table>	
							</div>
							}	
						</div>
					</div>
				</div>
				}
				
				<div class="modal fade" id="modalDialog" role="dialog">
					<div class="modal-dialog modal-lg">
					  <div class="modal-content">
						<div class="modal-header">
						  <button type="button" class="close" data-dismiss="modal">&times;</button>
						  <h2 class="modal-title text-center">Modal Header</h2>
						</div>
						<div class="modal-body"> 
						  <h4 class="text-center" id="kelas"></h4>
						  <h4 class="text-center" id="dosen"></h4>
						  <h4 class="text-center" id="waktu"></h4>
						  <h4 class="text-center" id="ruang"></h4>
						</div>
						<div class="modal-footer">
						  <button type="button" class="btn btn-default" data-dismiss="modal">Close</button>
						</div>
					  </div>
					</div>
				</div>
				
			</div>
		</div>
		
		<!-- Javascript -->
		<script src="@routes.Assets.versioned("javascripts/jquery-1.11.3.min.js")"></script>
		<script src="@routes.Assets.versioned("javascripts/script.js")"></script>
		<script src="@routes.Assets.versioned("bootstrap/js/bootstrap.min.js")"></script>
		<script>
			function setModal(kodeMk, namaMk, kelas, dosen, waktu, ruang) {
				$(".modal-title").html(kodeMk + " " + namaMk);
				$("#waktu").html("Pukul " + waktu);
				$("#kelas").html("Kelas " + kelas);
				$("#dosen").html(dosen);
				$("#ruang").html("Ruang " + ruang);
				//$(".modal-body").html("<table> <tr> <td>"+waktu+"</td></tr><tr><td>Kelas "+kelas+"</td></tr><tr><td>"+dosen+"</td></tr><tr><td>"+ruang+"</td></tr></table>");
			}
		</script>
    </body>
</html>
\end{lstlisting}

\singlespacing 
\begin{lstlisting}[language=html,basicstyle=\tiny,caption=ringkasan.scala.html]
@(ringkasan: models.display.RingkasanDisplay)

<!DOCTYPE html>

<html lang="en">
    <head>
        <meta charset="UTF-8">
		<meta http-equiv="X-UA-Compatible" content="IE=edge">
        <meta name="viewport" content="width=device-width, initial-scale=1">
        <link rel="icon" type="image/x-icon" href="@routes.Assets.versioned("images/logo-IT.png")" />
		<title>Informatika Student Portal</title>

		<!-- CSS -->
        <link rel="stylesheet" href="@routes.Assets.versioned("bootstrap/css/bootstrap.min.css")">
        <link rel="stylesheet" href="@routes.Assets.versioned("stylesheets/main.css")"> 
    </head>
    <body>
		<nav class="navbar navbar-inverse sidebar" role="navigation">
			<div class="container-fluid">
				<!-- Logo dan Toggle -->
				<div class="navbar-header">
					<button type="button" class="navbar-toggle" data-toggle="collapse" data-target="#sidebar-navbar-collapse">
						<span class="icon-bar"></span>
						<span class="icon-bar"></span>
						<span class="icon-bar"></span>
					</button>
					<div class="logo-set">
						<img src="@routes.Assets.versioned("images/logo-unpar.png")" width="25%" height="25%"/>
						<img src="@routes.Assets.versioned("images/logo-IT.png")" width="25%" height="25%"/>
					</div>
					<p class="navbar-brand">INFORMATIKA STUDENT PORTAL</p>
				</div>
				<!-- Menu Navigasi -->
				<div class="collapse navbar-collapse" id="sidebar-navbar-collapse">
					<ul class="nav navbar-nav">
						<li><a href="@routes.Application.home()">
							Home<span style="font-size:16px;" class="pull-right hidden-xs showopacity glyphicon glyphicon-home"></span>
						</a></li>
						<li><a href="@routes.Application.prasyarat()">
							Prasyarat Mata Kuliah<span style="font-size:16px;" class="pull-right hidden-xs showopacity glyphicon glyphicon-tasks"></span>
						</a></li>
						<li><a href="@routes.Application.jadwalKuliah()">
							 Jadwal Kuliah<span style="font-size:16px;" class="pull-right hidden-xs showopacity glyphicon glyphicon-calendar"></span>
						</a></li>
						<li class="active" ><a href="@routes.Application.ringkasan()">
							Data Akademik<span style="font-size:16px;" class="pull-right hidden-xs showopacity glyphicon glyphicon-book"></span>
						</a></li>
						<li ><a href="@routes.Application.logout()">
							Logout<span style="font-size:16px;" class="pull-right hidden-xs showopacity glyphicon glyphicon-log-out"></span>
						</a></li>
					</ul>
				</div>
			</div>
		</nav>
		<div class="main">
		<!-- Konten Halaman -->
			<div class="container-fluid">
				<div class="row">
					<h2 class="text-center">RINGKASAN DATA AKADEMIK</h2> 
				</div>
				<p></br></p>
				@if(ringkasan==null){
					<div class="row">
						<h5 style="color: gray;">DATA AKADEMIK BELUM TERSEDIA <span class="glyphicon glyphicon-exclamation-sign"></span></h5>
					</div>	
				}else{
				<div class="row">
					<div class="col-lg-2 ringkasan-panel"></div>
                    <div class="col-lg-8 ringkasan-panel">
						<div class="page-header">
							<h5 class="text-center">INDEKS PRESTASI</h5>
						</div>
						<div class="ringkasan-body">
							IPS @ringkasan.getSemesterTerakhir: @ringkasan.getIPS() dari @ringkasan.getSKSLulusSemTerakhir SKS<br>
							IPK: @ringkasan.getIPK()<br>
							SKS lulus: @ringkasan.getSKSLulusTotal()<br>
							Sisa SKS untuk kelulusan: @ringkasan.getMinSisaSKS()<br>
						</div>
					</div>
                </div>
				<p></br></p>
				<div class="row">
					<div class="col-lg-2 ringkasan-panel"></div>
                    <div class="col-lg-8 ringkasan-panel">
						<div class="page-header">
							<h5 class="text-center">PILIHAN WAJIB</h5>
						</div>
						<div class="ringkasan-body">
							Lulus: <br>
							<ul>
								@if(ringkasan.getPilWajibLulus().length==0){
									<li>	-	</li>
								}else{
									@for(list<-ringkasan.getPilWajibLulus()){
										<li>@list @ringkasan.getNamaPilWajib(list)</li>
									}
								}
							</ul>
							Belum lulus: <br>
							<ul>
								@if(ringkasan.getPilWajibBelumLulus().length==0){
									<li>	-	</li>
								}else{
									@for(list<-ringkasan.getPilWajibBelumLulus()){
										<li>@list @ringkasan.getNamaPilWajib(list)</li>
									}
								}
							</ul>
							<h5>LULUS @ringkasan.getPilWajibLulus().length DARI @ringkasan.getPilWajib().length MATA KULIAH PILIHAN WAJIB</h5>
							@if(ringkasan.getPilWajib().length-ringkasan.getPilWajibBelumLulus().length>=ringkasan.getMinLulusPilWajib()){
								<P style="color: #0066FF;">SUDAH MEMENUHI SYARAT KELULUSAN MATA KULIAH PILIHAN WAJIB <span class="glyphicon glyphicon-flag"></span></P>
							}else{
								<p style="color: red;">BELUM MEMENUHI SYARAT KELULUSAN MATA KULIAH PILIHAN WAJIB <span class="glyphicon glyphicon-remove"></span></p>
								<p style="color: #FF9900;">MINIMAL LULUS @(ringkasan.getMinLulusPilWajib()-ringkasan.getPilWajibLulus().length) MATA KULIAH PILIHAN WAJIB LAGI <span class="glyphicon glyphicon-warning-sign"></span></p>
							}
							</div>
					</div>
                </div>
                }
			</div>
		</div>
		<div>
		</div>
		<!-- Javascript -->
		<script src="@routes.Assets.versioned("javascripts/jquery-1.11.3.min.js")"></script>
		<script src="@routes.Assets.versioned("javascripts/script.js")"></script>
		<script src="@routes.Assets.versioned("bootstrap/js/bootstrap.min.js")"></script>
		
    </body>
</html>
\end{lstlisting}
