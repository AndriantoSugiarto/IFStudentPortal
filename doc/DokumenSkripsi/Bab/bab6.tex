\chapter{Kesimpulan dan Saran}
\label{chap:kesimpulan_saran}

\section{Kesimpulan}
\label{sec:kesimpulan}
Dari hasil pembangunan aplikasi Informatika Student Portal, didapatkanlah kesimpulan-kesimpulan sebagai berikut:
		\begin{enumerate}
			\item Fitur-fitur yang dibuat untuk Informatika Student Portal antara lain prasyarat mata kuliah, jadwal kuliah yang tersusun dan terurut berdasarkan hari, data akademik berupa IPS dan IPK yang langsung berubah ketika nilai sudah muncul, sisa SKS, dan status kelulusan mata kuliah pilihan wajib, dan aplikasi yang dapat diakses dari sistem operasi manapun.
			\item Telah berhasil mengimplementasikan \textit{web scraping} menggunakan \textit{library} jsoup. Dengan \textit{library} jsoup, data dari Portal Akademik Mahasiswa dapat diperoleh secara langsung.
			\item Aplikasi Informatika Student Portal telah berhasil dibangun dengan menggunakan Play Framework. Selain itu, aplikasi Informatika Student Portal juga memperoleh data dari Portal Akademik Mahasiswa secara langsung menggunakan \textit{library} jsoup kemudian mengolah data tersebut dengan bantuan SIA Models. 
		\end{enumerate}

\section{Saran}
\label{sec:saran}
Dari hasil penelitian termasuk kesimpulan yang didapat, berikut adalah beberapa saran untuk pengembangan:
	\begin{enumerate}
		\item Penelitian ini memanfaatkan SIA Model terutama \textit{package} ``id.ac.unpar.siamodels.matakuliah'' yang berisi aturan mengenai prasyarat mata kuliah. Aturan-aturan tersebut dapat berubah suatu saat. Oleh karena itu, sebaiknya SIA Models dipisahkan dari \textit{project} Play sehingga jika terjadi perubahan aturan pada SIA Models, tidak perlu dilakukan perubahan pada \textit{project} yang dapat mengakibatkan kesalahan kode program. 
		\item Dalam pengembangan berikutnya, perlu diperhatikan data dari mahasiswa yang pernah melakukan transfer studi. Jika memungkinkan, sebaiknya dianalisis langsung dari akun Portal Akademik Mahasiswa milik mahasiswa tersebut agar dapat mengambil data yang sesuai. Saat ini, aplikasi mengambil data langsung dari halaman riwayat nilai Portal Akademik Mahasiwa. Terdapat halaman lain yang memuat nilai mahasiswa pada Portal Akademik Mahasiswa yaitu halaman Daftar Perkembangan Studi (DPS). Sebaiknya pengambilan nilai dilakukan pada halaman DPS.
		\item Saat ini, apliaksi berjalan pada port 9000 dan tidak dijalankan secara otomatis saat server mulai berjalan. Sebaiknya aplikasi dapat langsung berjalan saat server mulai berjalan dan aplikasi juga dapat berjalan pada port 80.
		\item Aplikasi belum dapat menangani \textit{path} yang tidak terdapat pada \textit{routes} sehingga aplikasi akan memberikan pesan error jika memasuki \textit{path} tersebut. Dalam pengembangan berikutnya, path yang tidak ditemukan sebaiknya diarahkan ke halaman ``404 Not Found''.
	\end{enumerate}